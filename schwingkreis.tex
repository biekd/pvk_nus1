\section{Vorgehen: Resonanzfrequenz finden}

\begin{itemize}
\item [1.] Versuche alle Komponenten untereinander zu vereinfachen (Kondensatoren, Widerstände, Spulen). Denke daran, dass du Elemente in Serie vertauschen kannst.
\item [2.a] Hast du keine Zeit mehr oder handelt es sich um einen Standartschwingkreis, so ist $\omega_0 = \frac{1}{\sqrt{LC}}$ ein guter Ansatz.
\item [2.b] Versuche den Schwingkreis zu kategorisieren: Sind Kapazität und Spule in Serie $\rightarrow$ Serienschwingkreis, sonst $\rightarrow$ Parallelschwingkreis. \\
\item []\textbf{Serienschwingkreis}
\item[1.] Berechne Impedanz der Serienschaltung.
\item [2.] Die Resonanzfrequenz ist dann ereicht, wenn der Imaginärteil der Impedanz verschwindet: $Im\{Z_{serie}(\omega_x)\} = 0 \rightarrow \omega_0 = \omega_x$ \\
\item []\textbf{Parallelschwingkreis}
\item [1.]Berechne die Admitanz der Parallelschaltung.
\item [2.] Die Resonanzfrequenz ist dann ereicht, wenn der Imaginärteil der Admitanz verschwindet: $Im\{Y_{parallel}(\omega_x)\} = 0 \rightarrow \omega_0 = \omega_x$ \\

\item []\textbf{Keine Kategorie}
\item [1.] Versuche eine Spannung über einer seriellen Kapazität/Spule $\underline{U_{c/l}}$ oder den Strom $\underline{I_{c/l}}$ durch eine parallele Kapazität/Spule zu finden.
\item [2.] Leite diese Grösse nach Omega ab und setze sie zu 0. $\frac{dU}{d\omega} (\omega_x) = 0 / \frac{dI}{d\omega} (\omega_x) = 0$
\end{itemize}

%%%%%%%%%%%%%%%%%%%%%%%%%%%%%%%%%%%%%%%%%
% Structured General Purpose Assignment
% LaTeX Template
%
% This template has been downloaded from:
% http://www.latextemplates.com
%
% Original author:
% Ted Pavlic (http://www.tedpavlic.com)
%
% Note:
% The \lipsum[#] commands throughout this template generate dummy text
% to fill the template out. These commands should all be removed when
% writing assignment content.
%
%%%%%%%%%%%%%%%%%%%%%%%%%%%%%%%%%%%%%%%%%

%----------------------------------------------------------------------------------------
%	PACKAGES AND OTHER DOCUMENT CONFIGURATIONS
%----------------------------------------------------------------------------------------

\documentclass{article}

\usepackage{graphicx,wrapfig,lipsum}
\usepackage{fancyhdr} % Required for custom headers
\usepackage{lastpage} % Required to determine the last page for the footer
\usepackage{extramarks} % Required for headers and footers
\usepackage{graphicx} % Required to insert images
\usepackage{lipsum} % Used for inserting dummy 'Lorem ipsum' text into the template

% Margins
\topmargin=-0.45in
\evensidemargin=0in
\oddsidemargin=0in
\textwidth=6.5in
\textheight=9.0in
\headsep=0.25in

\linespread{1.1} % Line spacing

% Set up the header and footer
\pagestyle{fancy}
\lhead{\hmwkAuthorName} % Top left header
\chead{\hmwkTitle} % Top center header
\rhead{\firstxmark} % Top right header
\lfoot{\lastxmark} % Bottom left footer
\cfoot{} % Bottom center footer
\rfoot{Page\ \thepage\ of\ \pageref{LastPage}} % Bottom right footer
\renewcommand\headrulewidth{0.4pt} % Size of the header rule
\renewcommand\footrulewidth{0.4pt} % Size of the footer rule

\setlength\parindent{0pt} % Removes all indentation from paragraphs

%----------------------------------------------------------------------------------------
%	DOCUMENT STRUCTURE COMMANDS
%	Skip this unless you know what you're doing
%----------------------------------------------------------------------------------------

% Header and footer for when a page split occurs within a problem environment
\newcommand{\enterProblemHeader}[1]{
\nobreak\extramarks{#1}{#1 continued on next page\ldots}\nobreak
\nobreak\extramarks{#1 (continued)}{#1 continued on next page\ldots}\nobreak
}

% Header and footer for when a page split occurs between problem environments
\newcommand{\exitProblemHeader}[1]{
\nobreak\extramarks{#1 (continued)}{#1 continued on next page\ldots}\nobreak
\nobreak\extramarks{#1}{}\nobreak
}

\setcounter{secnumdepth}{0} % Removes default section numbers
\newcounter{homeworkProblemCounter} % Creates a counter to keep track of the number of problems

\newcommand{\homeworkProblemName}{}
\newenvironment{homeworkProblem}[1][Netzwerke und Schaltungen 1]{ % Makes a new environment called homeworkProblem which takes 1 argument (custom name) but the default is "Problem #"
\stepcounter{homeworkProblemCounter} % Increase counter for number of problems
\renewcommand{\homeworkProblemName}{#1} % Assign \homeworkProblemName the name of the problem
\section{\homeworkProblemName} % Make a section in the document with the custom problem count
\enterProblemHeader{\homeworkProblemName} % Header and footer within the environment
}{
\exitProblemHeader{\homeworkProblemName} % Header and footer after the environment
}

\newcommand{\problemAnswer}[1]{ % Defines the problem answer command with the content as the only argument
\noindent\framebox[\columnwidth][c]{\begin{minipage}{0.98\columnwidth}#1\end{minipage}} % Makes the box around the problem answer and puts the content inside
}

\newcommand{\homeworkSectionName}{}
\newenvironment{homeworkSection}[1]{ % New environment for sections within homework problems, takes 1 argument - the name of the section
\renewcommand{\homeworkSectionName}{#1} % Assign \homeworkSectionName to the name of the section from the environment argument
\subsection{\homeworkSectionName} % Make a subsection with the custom name of the subsection
\enterProblemHeader{\homeworkProblemName\ [\homeworkSectionName]} % Header and footer within the environment
}{
\enterProblemHeader{\homeworkProblemName} % Header and footer after the environment
}

%----------------------------------------------------------------------------------------
%	NAME AND CLASS SECTION
%----------------------------------------------------------------------------------------

\newcommand{\hmwkTitle}{Netzwerk und Schaltungen 1 - Prüfungskatalog} % Assignment title
\newcommand{\hmwkDueDate}{Monday,\ January\ 1,\ 2012} % Due date
\newcommand{\hmwkClass}{NuS 1} % Course/class
\newcommand{\hmwkClassTime}{} % Class/lecture time
\newcommand{\hmwkClassInstructor}{} % Teacher/lecturer
\newcommand{\hmwkAuthorName}{Ren\'e Zurbr\"ugg} % Your name
\usepackage{amsmath}
\usepackage{esint}
%----------------------------------------------------------------------------------------
%	TITLE PAGE
%----------------------------------------------------------------------------------------

%----------------------------------------------------------------------------------------

\begin{document}


%----------------------------------------------------------------------------------------
%	PROBLEM 1
%----------------------------------------------------------------------------------------

% To have just one problem per page, simply put a \clearpage after each problem

\begin{homeworkProblem}



a) Es gilt:
 \begin{center}
  $\iint _A \vec{J}(\vec{r}) \cdot d\vec{A} = I$.
\end{center}
Falls das J-Feld und die Fläche \textbf{senkrecht} sind und das J-Feld überall auf dieser Fläche  \textbf{gleich Gross} ist, vereinfacht sich dies zu:
\begin{center}
  $ A_{eff}(\vec{r}) \cdot J(\vec{r})  = I \Rightarrow J(\vec{r}) = \frac{I}{A_{eff}(\vec{r})}$ \\
\end{center}
Wobei $A_{eff}$ die effektiv vom Strom durchflossene Fläche bezeichnet.
Somit gilt für die Stromdichte im Messwiderstand:
\begin{center}
  $ J(r) = \frac{I}{d \cdot 2 \pi r}$ \\
\end{center}

und somit in Vektorform (Zylinderkoordinaten):
\begin{center}
  $ \vec{J}(r) = \frac{I}{d \cdot 2 \pi r} \cdot \vec{e}_{r}$ \\
\end{center}

 b) Der Zusammenhang zwischen E-Feld un J-Feld ist gegeben als:
\begin{center}
  $\vec{E} = \frac{1}{\kappa} \cdot \vec{J}$
\end{center}
Somt gilt für das E-Feld:
\begin{center}
    $\vec{E}(r) =  \frac{I}{d \cdot 2 \pi r \cdot \kappa} \cdot \vec{e}_{r}$
\end{center}
Für die Spannung $U_{AB}$ gilt: \\
\begin{center}
  $U_{AB} = \int_A^B \vec{E}(r) \cdot d\vec{s} = \int_{\frac{D_{Innen}}{2}}^{\frac{D_{Aussen}}{2}} E(r) \cdot dr =  \int_{\frac{D_{Innen}}{2}}^{\frac{D_{Aussen}}{2}} \frac{I}{d \cdot 2 \pi r \cdot \kappa} \cdot dr = \underline{\underline{\frac{I}{\kappa \cdot d \cdot 2 \pi} \cdot ln(\frac{D_{aussen}}{D_{innen}})}}$
\end{center}
Für den Widerstand R gilt:
\begin{center}
$  R_{AB} := \frac{U_{AB}}{I} = \frac{1}{\kappa \cdot d \cdot 2 \pi} \cdot ln(\frac{D_{aussen}}{D_{innen}})$
\end{center}

Mit den Werten: $D_{aussen} = 2cm$, $D_{innen} = 5mm$ , $d = 3mm$ und $\kappa = 12 \cdot 10^3 \frac{S}{m}$ gilt:
\begin{center}
  $R_{AB} = \frac{1}{12 \cdot 10^3 \frac{S}{m} \cdot 3\cdot 10^{-3} m \cdot 2 \pi} \cdot ln(\frac{2 \cdot 10^{-2}}{3 \cdot 10^{-3}}) = 8.387m\Omega$
\end{center}
 c) Wir betrachten beide Fälle (-3mm und +3mm):  \\
 \begin{center}
 +3mm: $\rightarrow R' = \frac{1}{12 \cdot 10^3 \frac{S}{m} \cdot 3\cdot 10^{-3} m \cdot 2 \pi} \cdot ln(\frac{2 \cdot 10^{-2} + 3 \cdot 10^{-3}}{3 \cdot 10^{-3}}) = 9.005m\Omega \rightarrow \Delta R =  |R-R'| = 0.618 m\Omega$\\
 -3mm: $\rightarrow R' = \frac{1}{12 \cdot 10^3 \frac{S}{m} \cdot 3\cdot 10^{-3} m \cdot 2 \pi} \cdot ln(\frac{2 \cdot 10^{-2} - 3 \cdot 10^{-3}}{3 \cdot 10^{-3}}) = 7.669m\Omega \rightarrow \Delta R =  |R-R'| = 1.336 m\Omega$\\
 Daraus Folgt: Maximaler Fehler bei $-3mm$. $R'$ ist dann $7.669m\Omega$ und für den Fehler gilt: $ \Delta R = 1.336 m\Omega$
\end{center}


 d) Beim Messen gilt:
\begin{center}
 $\frac{U_{AB}}{R_{AB}} = I \rightarrow \Delta I =  |\frac{U_{AB}}{R_{AB}} - \frac{U_{AB}}{R'_{AB}} | $ \\
 $ F = \frac {\Delta I}{I} = \frac{\Delta I}{U_{AB}/R_{AB}} = |1 - \frac{R_{AB}}{R'_{AB}}| = |1 - \frac{8.387m\Omega}{ 7.669}| = 9.28\% $
\end{center}


 e) es gilt: $ U_{AC} = \int_{r_a}^{r_c} E \cdot dr$ und  $ U_{CB} = \int_{r_c}^{r_b} E \cdot dr$.  \\
Das Integral $\int_{r_a}^{r_b} E \cdot dr$ haben wir bereits ausgerechnet. Es ergibt:
$ \frac{I}{\kappa \cdot d \cdot 2 \pi} \cdot ln(\frac{r_b}{r_a}) $ \\
Somit lautet die Gleichung:
\begin{center}
  $ \frac{I}{\kappa \cdot d \cdot 2 \pi} \cdot ln(\frac{r_c}{\frac{D_{innen}}{2} }) = \frac{I}{\kappa \cdot d \cdot 2 \pi} \cdot ln(\frac{\frac{D_{aussen}}{2}}{r_c })$ \\
  $ \Rightarrow \frac{2 r_c}{D_{innen}} = \frac{D_{aussen}}{2 r_c} \rightarrow r_c = \frac{\sqrt{D_{aussen} \cdot D_{innen}}}{2} = 5mm$
\end{center}




\end{homeworkProblem}



%----------------------------------------------------------------------------------------


\end{document}

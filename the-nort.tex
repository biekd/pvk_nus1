\section{Vorgehen: Thévenin / Norton Äquivalent}

\begin{itemize}

  \item[1.] Betrachte die Schaltung. Sind mehr Stromquellen vorhanden, so ist es häufig einfacher den Kurzschlussstrom zu berechnen. Bei mehr Spannungsquellen die Leerlaufsspannung. \\
  Wiederhole folgendes für alle Quellen $i$:

  \end{itemize}
  \beginip
  \begin{itemize}

  \item [2.  ]  Setze alle Strom / Spannungsquellen ausser einer zu 0. (Stromquelle $\rightarrow$ Leerlauf, Spannungsquelle $\rightarrow$ Kurzschluss)
  \item[3. ] Versuche die einzelne Quelle auf die linke Seite zu bekommen. (Siehe Skript \texttt{"}Flussverfahren\texttt{"})
  \item[4.a)] \textbf{Kurzschlussstrom}
  \begin{enumerate}
  \item Schliesse die Klemmen kurz und bezeichne den Strom, welcher durch diese Klemmen fliesst als $I_{ks}^{(i)}$.
  \item Versuche mittels Stromteilern den gesuchten Strom zu berechnen. Falls eine direkte Verbindung von Stromquelle über den Kurzschluss zur Quelle zurückfürt, ist der Kurzschlussstrom gleich dem Strom der Quelle (=0 bei Spannungsquelle). \\
  Ist die Quelle keine Stromquelle, so kann evt. ein serieller Widerstand verwendet werden, um die Quelle umzuformen.
  \end{enumerate}
  \item[4.b)] \textbf{Leerlaufspannung}
  \begin{enumerate}
  \item Zeichnen einen Spannungspfeil zwischen den Klemmen und bezeichne die Spannung als $U_{LL}^{(i)}$.
  \item Versuche mittels Spannungsteiler die gesuchte Spannung zu berechnen. Falls kein Strom von der Spannungsquelle fliessen kann (Leerlauf unterbricht die komplette Schaltung), so ist die Leerlaufspannung gleich der Spannung der Spannungsquelle (=0 bei Stromquelle). \\
  Ist die Quelle keine Spannungsquelle, so kann evt. ein paralleler Widerstand verwendet werden, um die Quelle umzuformen.
  \end{enumerate}
\end{itemize}

  \iend
\begin{itemize}

  \item[5] \textbf{Innenwiderstand}
  \begin{enumerate}
    \item Setze \textbf{alle} Quellen auf 0. Versuche nun die Widerstände solange umzuformen, bis nur noch ein Ersatzwiderstand vorhanden ist. (Ggf. \texttt{"}Flussverfahren\texttt{"} mit offener Klemme anwenden)
    \item Bezeichne den Wert des Widerstandes als $R_i$
  \end{enumerate}
  \item[6.a)] \textbf{Thévenin Äquivalent} Spannungsquelle mit seriellem Innenwiderstand. \\
  Werte: $\displaystyle   R= R_i$, $U_q = \sum_i U_{LL}^{(i)} = (\sum_i I_{ks}^{(i)})\cdot R_i$

  \item[6.b)] \textbf{Norton Äquivalent} Spannungsquelle mit parallelem Innenwiderstand. \\
  Werte: $\displaystyle  R= R_i$, $I_q = \sum_i I_{ks}^{(i)}) = \frac{\sum_i U_{LL}^{(i)}}{R_i}$

  \end{itemize}

\section{Vorgehen: Leistungsanpassung}

\textbf{Falls möglich} \\
\begin{itemize}
  \item [1. ] Entferne Komponenten über denen die Leistung maximiert werden soll.
  \item [2. ] Berechne von den Klemmen den Innenwiderstand.
  \item [3.a] \textbf{Realer Innenwiderstand} $R_i \rightarrow R_L = R_i$
  \item [3.b] \textbf{Komplexer Innenwiderstand} $R_i + j \cdot Z_i  \rightarrow R_L + Z_L = R_i - j \cdot Z_i$
  \item [4] Falls maximale Leistung gefragt: Berechne Leerlaufpsannung oder den Kurzschlussstrom. $P_{max} = \frac{U_{LL}^2}{4\cdot R_i}$
\end{itemize}
\textbf{Sonst} \\

\begin{itemize}
  \item [1. ] Finde einen Ausdruck für $\underline{U_L}$ und $\underline{I_L}$
  \item [2. ] Berechne $\underline{P} =\underline{U_L}\cdot\underline{I_L^*}$
  \item [3. ] Leite $Re\{P_L\}$ nach $Re\{Z_L\}$ und $Im\{P_L\}$ nach $Im\{Z_L\}$ ab und setze zu 0.
\end{itemize}

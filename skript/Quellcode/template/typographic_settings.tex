%%%% Time-stamp: <2015-08-22 17:20:32 vk>
%%%% === Disclaimer: =======================================================
%% created by
%%
%%      Karl Voit
%%
%% using GNU/Linux, GNU Emacs & LaTeX 2e
%%
%doc%
%doc% \section{\texttt{typographic\_settings.tex} --- Typographic finetuning}
%doc%
%doc% The settings of file \verb#template/typographic_settings.tex# contain
%doc% typographic finetuning related to things mentioned in literature.  The
%doc% settings in this file relates to personal taste and most of all:
%doc% \emph{typographic experience}.
%doc%
%doc% \paragraph{What should I do with this file?} You might as well skip the whole
%doc% file by excluding the \verb#%%%% Time-stamp: <2015-08-22 17:20:32 vk>
%%%% === Disclaimer: =======================================================
%% created by
%%
%%      Karl Voit
%%
%% using GNU/Linux, GNU Emacs & LaTeX 2e
%%
%doc%
%doc% \section{\texttt{typographic\_settings.tex} --- Typographic finetuning}
%doc%
%doc% The settings of file \verb#template/typographic_settings.tex# contain
%doc% typographic finetuning related to things mentioned in literature.  The
%doc% settings in this file relates to personal taste and most of all: 
%doc% \emph{typographic experience}. 
%doc% 
%doc% \paragraph{What should I do with this file?} You might as well skip the whole
%doc% file by excluding the \verb#%%%% Time-stamp: <2015-08-22 17:20:32 vk>
%%%% === Disclaimer: =======================================================
%% created by
%%
%%      Karl Voit
%%
%% using GNU/Linux, GNU Emacs & LaTeX 2e
%%
%doc%
%doc% \section{\texttt{typographic\_settings.tex} --- Typographic finetuning}
%doc%
%doc% The settings of file \verb#template/typographic_settings.tex# contain
%doc% typographic finetuning related to things mentioned in literature.  The
%doc% settings in this file relates to personal taste and most of all: 
%doc% \emph{typographic experience}. 
%doc% 
%doc% \paragraph{What should I do with this file?} You might as well skip the whole
%doc% file by excluding the \verb#%%%% Time-stamp: <2015-08-22 17:20:32 vk>
%%%% === Disclaimer: =======================================================
%% created by
%%
%%      Karl Voit
%%
%% using GNU/Linux, GNU Emacs & LaTeX 2e
%%
%doc%
%doc% \section{\texttt{typographic\_settings.tex} --- Typographic finetuning}
%doc%
%doc% The settings of file \verb#template/typographic_settings.tex# contain
%doc% typographic finetuning related to things mentioned in literature.  The
%doc% settings in this file relates to personal taste and most of all: 
%doc% \emph{typographic experience}. 
%doc% 
%doc% \paragraph{What should I do with this file?} You might as well skip the whole
%doc% file by excluding the \verb#\input{template/typographic_settings.tex}# command
%doc% in \texttt{main.tex}.  For standard usage it is recommended to stay with the
%doc% default settings.
%doc% 
%doc% 
%% ========================================================================

%doc%
%doc% Some basic microtypographic settings are provided by the
%doc% \texttt{microtype}
%doc% package\footnote{\url{http://ctan.org/pkg/microtype}}. This template
%doc% uses the rather conservative package parameters: \texttt{protrusion=true,factor=900}.
\usepackage[protrusion=true,factor=900]{microtype}

%doc%
%doc% \subsection{French spacing}
%doc%
%doc% \paragraph{Why?} see~\textcite[p.\,28, p.\,30]{Bringhurst1993}: `2.1.4 Use a single word space between sentences.'
%doc%
%doc% \paragraph{How?} see~\textcite[p.\,185]{Eijkhout2008}:\\
%doc% \verb#\frenchspacing  %% Macro to switch off extra space after punctuation.# \\
\frenchspacing  %% Macro to switch off extra space after punctuation.
%doc%
%doc% Note: This setting might be default for \myacro{KOMA} script.
%doc%


%doc%
%doc% \subsection{Font}
%doc% 
%doc% This template is using the Palatino font (package \texttt{mathpazo}) which results
%doc% in a legible document and matching mathematical fonts for printout.
%doc% 
%doc% It is highly recommended that you either stick to the Palatino font or use the
%doc% \LaTeX{} default fonts (by removing the package \texttt{mathpazo}).
%doc% 
%doc% Chosing different fonts is not
%doc% an easy task. Please leave this to people with good knowledge on this subject.
%doc% 
%doc% One valid reason to change the default fonts is when your document is mainly
%doc% read on a computer screen. In this case it is recommended to switch to a font
%doc% \textsf{which is sans-serif like this}. This template contains several alternative
%doc% font packages which can be activated in this file.
%doc% 

% for changing the default font, please go to the next subsection!

%doc%
%doc% \subsection{Text figures}
%doc% 
%doc% \ldots also called old style numbers such as 0123456789. 
%doc% (German: \enquote{Mediäval\-ziffern\footnote{\url{https://secure.wikimedia.org/wikibooks/de/wiki/LaTeX-W\%C3\%B6rterbuch:\_Medi\%C3\%A4valziffern}}})
%doc% \paragraph{Why?} see~\textcite[p.\,44f]{Bringhurst1993}: 
%doc% \begin{quote}
%doc% `3.2.1 If the font includes both text figures and titling figures, use
%doc%  titling figures only with full caps, and text figures in all other
%doc%  circumstances.'
%doc% \end{quote}
%doc% 
%doc% \paragraph{How?} 
%doc% Quoted from Wikibooks\footnote{\url{https://secure.wikimedia.org/wikibooks/en/wiki/LaTeX/Formatting\#Text\_figures\_.28.22old\_style.22\_numerals.29}}:
%doc% \begin{quote}
%doc% Some fonts do not have text figures built in; the textcomp package attempts to
%doc% remedy this by effectively generating text figures from the currently-selected
%doc% font. Put \verb#\usepackage{textcomp}# in your preamble. textcomp also allows you to
%doc% use decimal points, properly formatted dollar signs, etc. within
%doc% \verb#\oldstylenums{}#.
%doc% \end{quote}
%doc% \ldots but proposed \LaTeX{} method does not work out well. Instead use:\\
%doc% \verb#\usepackage{hfoldsty}#  (enables text figures using additional font) or \\
%doc% \verb#\usepackage[sc,osf]{mathpazo}# (switches to Palatino font with small caps and old style figures enabled).
%doc%
%\usepackage{hfoldsty}  %% enables text figures using additional font
%% ... OR use ...
\usepackage[sc,osf]{mathpazo} %% switches to Palatino with small caps and old style figures

%% Font selection from:
%%     http://www.matthiaspospiech.de/latex/vorlagen/allgemein/preambel/fonts/
%% use following lines *instead* of the mathpazo package above:
%% ===== Serif =========================================================
%% for Computer Modern (LaTeX default font), simply remove the mathpazo above
%\usepackage{charter}\linespread{1.05} %% Charter
%\usepackage{bookman}                  %% Bookman (laedt Avant Garde !!)
%\usepackage{newcent}                  %% New Century Schoolbook (laedt Avant Garde !!)
%% ===== Sans Serif ====================================================
%\renewcommand{\familydefault}{\sfdefault}  %% this one in *combination* with the default mathpazo package
%\usepackage{cmbright}                  %% CM-Bright (eigntlich eine Familie)
%\usepackage{tpslifonts}                %% tpslifonts % Font for Slides


%doc% 
%doc% \subsection{\texttt{myacro} --- Abbrevations using \textsc{small caps}}\myinteresting
%doc% \label{sec:myacro}
%doc% 
%doc% \paragraph{Why?} see~\textcite[p.\,45f]{Bringhurst1993}: `3.2.2 For abbrevations and
%doc% acronyms in the midst of normal text, use spaced small caps.'
%doc% 
%doc% \paragraph{How?} Using the predefined macro \verb#\myacro{}# for things like
%doc% \myacro{UNO} or \myacro{UNESCO} using \verb#\myacro{UNO}# or \verb#\myacro{UNESCO}#.
%doc% 
\DeclareRobustCommand{\myacro}[1]{\textsc{\lowercase{#1}}} %%  abbrevations using small caps


%doc% 
%doc% \subsection{Colorized headings and links}
%doc% 
%doc% This document template is able to generate an output that uses colorized
%doc% headings, captions, page numbers, and links. The color named `DispositionColor'
%doc% used in this document is defined near the definition of package \texttt{color}
%doc% in the preamble (see section~\ref{subsec:miscpackages}). The changes required
%doc% for headings, page numbers, and captions are defined here.
%doc% 
%doc% Settings for colored links are handled by the definitions of the
%doc% \texttt{hyperref} package (see section~\ref{sec:pdf}).
%doc% 
\setheadsepline{.4pt}[\color{DispositionColor}]
\renewcommand{\headfont}{\normalfont\sffamily\color{DispositionColor}}
\renewcommand{\pnumfont}{\normalfont\sffamily\color{DispositionColor}}
\addtokomafont{disposition}{\color{DispositionColor}}
\addtokomafont{caption}{\color{DispositionColor}\footnotesize}
\addtokomafont{captionlabel}{\color{DispositionColor}}

%doc% 
%doc% \subsection{No figures or tables below footnotes}
%doc% 
%doc% \LaTeX{} places floating environments below footnotes if \texttt{b}
%doc% (bottom) is used as (default) placement algorithm. This is certainly
%doc% not appealing for most people and is deactivated in this template by
%doc% using the package \texttt{footmisc} with its option \texttt{bottom}.
%doc% 
%% see also: http://www.komascript.de/node/858 (German description)
\usepackage[bottom]{footmisc}

%doc% 
%doc% \subsection{Spacings of list environments}
%doc% 
%doc% By default, \LaTeX{} is using vertical spaces between items of enumerate, 
%doc% itemize and description environments. This is fine for multi-line items.
%doc% Many times, the user does just write single-line items where the larger
%doc% vertical space is inappropriate. The \href{http://ctan.org/pkg/enumitem}{enumitem}
%doc% package provides replacements for the pre-defined list environments and
%doc% offers many options to modify their appearances.
%doc% This template is using the package option for \texttt{noitemsep} which
%doc% mimimizes the vertical space between list items.
%doc% 
\usepackage{enumitem}
\setlist{noitemsep}   %% kills the space between items

%doc% 
%doc% \subsection{\texttt{csquotes} --- Correct quotation marks}\myinteresting
%doc% \label{sub:csquotes}
%doc% 
%doc% \emph{Never} use quotation marks found on your keyboard.
%doc% They end up in strange characters or false looking quotation marks.
%doc% 
%doc% In \LaTeX{} you are able to use typographically correct quotation marks. The package 
%doc% \href{http://www.ctan.org/pkg/csquotes}{\texttt{csquotes}} offers you with 
%doc% \verb#\enquote{foobar}# a command to get correct quotation marks around \enquote{foobar}.
%doc% Please do check the package options in order to modify
%doc% its settings according to the language used\footnote{most of the time in 
%doc% combination with the language set in the options of the \texttt{babel} package}.
%doc% 
%doc% \href{http://www.ctan.org/pkg/csquotes}{\texttt{csquotes}} is also recommended 
%doc% by \texttt{biblatex} (see Section~\ref{sec:references}). 
\usepackage[babel=true,strict=true,english=american,german=guillemets]{csquotes}

%doc% 
%doc% \subsection{Line spread}
%doc% 
%doc% If you have to enlarge the distance between two lines of text, you can
%doc% increase it using the \texttt{\mylinespread} command in \texttt{main.tex}. By default, it is
%doc% deactivated (set to 100~percent). Modify only with caution since it influences the
%doc% page layout and could lead to ugly looking documents.
\linespread{\mylinespread}

%doc% 
%doc% \subsection{Optional: Lines above and below the chapter head}
%doc% 
%doc% This is not quite something typographic but rather a matter of taste.
%doc% \myacro{KOMA} Script offers \href{http://www.komascript.de/node/24}{a method to
%doc% add lines above and below chapter head} which is disabled by
%doc% default. If you want to enable this feature, remove corresponding
%doc% comment characters from the settings.
%doc% 
%% Source: http://www.komascript.de/node/24
%disabled% %% 1st get a new command
%disabled% \newcommand*{\ORIGchapterheadstartvskip}{}%
%disabled% %% 2nd save the original definition to the new command
%disabled% \let\ORIGchapterheadstartvskip=\chapterheadstartvskip
%disabled% %% 3rd redefine the command using the saved original command
%disabled% \renewcommand*{\chapterheadstartvskip}{%
%disabled%   \ORIGchapterheadstartvskip
%disabled%   {%
%disabled%     \setlength{\parskip}{0pt}%
%disabled%     \noindent\color{DispositionColor}\rule[.3\baselineskip]{\linewidth}{1pt}\par
%disabled%   }%
%disabled% }
%disabled% %% see above
%disabled% \newcommand*{\ORIGchapterheadendvskip}{}%
%disabled% \let\ORIGchapterheadendvskip=\chapterheadendvskip
%disabled% \renewcommand*{\chapterheadendvskip}{%
%disabled%   {%
%disabled%     \setlength{\parskip}{0pt}%
%disabled%     \noindent\color{DispositionColor}\rule[.3\baselineskip]{\linewidth}{1pt}\par
%disabled%   }%
%disabled%   \ORIGchapterheadendvskip
%disabled% }

%doc% 
%doc% \subsection{Optional: Chapter thumbs}
%doc% 
%doc% This is not quite something typographic but rather a matter of taste.
%doc% \myacro{KOMA} Script offers \href{http://www.komascript.de/chapterthumbs-example}{a method to
%doc% add chapter thumbs} (in combination with the package \texttt{scrpage2}) which is disabled by
%doc% default. If you want to enable this feature, remove corresponding
%doc% comment characters from the settings.
%doc% 
%disabled% \makeatletter
%disabled% % Safty first
%disabled% \@ifundefined{chapter}{\let\chapter\undefined
%disabled%   \chapter must be defined to use chapter thumbs!}{%
%disabled%  
%disabled% % Two new commands for the width and height of the boxes with the
%disabled% % chapter number at the thumbs (use of commands instead of lengths
%disabled% % for sparing registers)
%disabled% \newcommand*{\chapterthumbwidth}{2em}
%disabled% \newcommand*{\chapterthumbheight}{1em}
%disabled%  
%disabled% % Two new commands for the colors of the box background and the
%disabled% % chapter numbers of the thumbs
%disabled% \newcommand*{\chapterthumbboxcolor}{black}
%disabled% \newcommand*{\chapterthumbtextcolor}{white}
%disabled%  
%disabled% % New command to set a chapter thumb. I'm using a group at this
%disabled% % command, because I'm changing the temporary dimension \@tempdima
%disabled% \newcommand*{\putchapterthumb}{%
%disabled%   \begingroup
%disabled%     \Large
%disabled%     % calculate the horizontal possition of the right paper border
%disabled%     % (I ignore \hoffset, because I interprete \hoffset moves the page
%disabled%     % at the paper e.g. if you are using cropmarks)
%disabled%     \setlength{\@tempdima}{\@oddheadshift}% (internal from scrpage2)
%disabled%     \setlength{\@tempdima}{-\@tempdima}%
%disabled%     \addtolength{\@tempdima}{\paperwidth}%
%disabled%     \addtolength{\@tempdima}{-\oddsidemargin}%
%disabled%     \addtolength{\@tempdima}{-1in}%
%disabled%     % putting the thumbs should not change the horizontal
%disabled%     % possition
%disabled%     \rlap{%
%disabled%       % move to the calculated horizontal possition
%disabled%       \hspace*{\@tempdima}%
%disabled%       % putting the thumbs should not change the vertical
%disabled%       % possition
%disabled%       \vbox to 0pt{%
%disabled%         % calculate the vertical possition of the thumbs (I ignore
%disabled%         % \voffset for the same reasons told above)
%disabled%         \setlength{\@tempdima}{\chapterthumbwidth}%
%disabled%         \multiply\@tempdima by\value{chapter}%
%disabled%         \addtolength{\@tempdima}{-\chapterthumbwidth}%
%disabled%         \addtolength{\@tempdima}{-\baselineskip}%
%disabled%         % move to the calculated vertical possition
%disabled%         \vspace*{\@tempdima}%
%disabled%         % put the thumbs left so the current horizontal possition
%disabled%         \llap{%
%disabled%           % and rotate them
%disabled%           \rotatebox{90}{\colorbox{\chapterthumbboxcolor}{%
%disabled%               \parbox[c][\chapterthumbheight][c]{\chapterthumbwidth}{%
%disabled%                 \centering
%disabled%                 \textcolor{\chapterthumbtextcolor}{%
%disabled%                   \strut\thechapter}\\
%disabled%               }%
%disabled%             }%
%disabled%           }%
%disabled%         }%
%disabled%         % avoid overfull \vbox messages
%disabled%         \vss
%disabled%       }%
%disabled%     }%
%disabled%   \endgroup
%disabled% }
%disabled%  
%disabled% % New command, which works like \lohead but also puts the thumbs (you
%disabled% % cannot use \ihead with this definition but you may change this, if
%disabled% % you use more internal scrpage2 commands)
%disabled% \newcommand*{\loheadwithchapterthumbs}[2][]{%
%disabled%   \lohead[\putchapterthumb#1]{\putchapterthumb#2}%
%disabled% }
%disabled%  
%disabled% % initial use
%disabled% \loheadwithchapterthumbs{}
%disabled% \pagestyle{scrheadings}
%disabled%  
%disabled% }
%disabled% \makeatother

%%%% END
%%% Local Variables:
%%% mode: latex
%%% mode: auto-fill
%%% mode: flyspell
%%% eval: (ispell-change-dictionary "en_US")
%%% TeX-master: "../main"
%%% End:
%% vim:foldmethod=expr
%% vim:fde=getline(v\:lnum)=~'^%%%%'?0\:getline(v\:lnum)=~'^%doc.*\ .\\%(sub\\)\\?section{.\\+'?'>1'\:'1':
# command
%doc% in \texttt{main.tex}.  For standard usage it is recommended to stay with the
%doc% default settings.
%doc% 
%doc% 
%% ========================================================================

%doc%
%doc% Some basic microtypographic settings are provided by the
%doc% \texttt{microtype}
%doc% package\footnote{\url{http://ctan.org/pkg/microtype}}. This template
%doc% uses the rather conservative package parameters: \texttt{protrusion=true,factor=900}.
\usepackage[protrusion=true,factor=900]{microtype}

%doc%
%doc% \subsection{French spacing}
%doc%
%doc% \paragraph{Why?} see~\textcite[p.\,28, p.\,30]{Bringhurst1993}: `2.1.4 Use a single word space between sentences.'
%doc%
%doc% \paragraph{How?} see~\textcite[p.\,185]{Eijkhout2008}:\\
%doc% \verb#\frenchspacing  %% Macro to switch off extra space after punctuation.# \\
\frenchspacing  %% Macro to switch off extra space after punctuation.
%doc%
%doc% Note: This setting might be default for \myacro{KOMA} script.
%doc%


%doc%
%doc% \subsection{Font}
%doc% 
%doc% This template is using the Palatino font (package \texttt{mathpazo}) which results
%doc% in a legible document and matching mathematical fonts for printout.
%doc% 
%doc% It is highly recommended that you either stick to the Palatino font or use the
%doc% \LaTeX{} default fonts (by removing the package \texttt{mathpazo}).
%doc% 
%doc% Chosing different fonts is not
%doc% an easy task. Please leave this to people with good knowledge on this subject.
%doc% 
%doc% One valid reason to change the default fonts is when your document is mainly
%doc% read on a computer screen. In this case it is recommended to switch to a font
%doc% \textsf{which is sans-serif like this}. This template contains several alternative
%doc% font packages which can be activated in this file.
%doc% 

% for changing the default font, please go to the next subsection!

%doc%
%doc% \subsection{Text figures}
%doc% 
%doc% \ldots also called old style numbers such as 0123456789. 
%doc% (German: \enquote{Mediäval\-ziffern\footnote{\url{https://secure.wikimedia.org/wikibooks/de/wiki/LaTeX-W\%C3\%B6rterbuch:\_Medi\%C3\%A4valziffern}}})
%doc% \paragraph{Why?} see~\textcite[p.\,44f]{Bringhurst1993}: 
%doc% \begin{quote}
%doc% `3.2.1 If the font includes both text figures and titling figures, use
%doc%  titling figures only with full caps, and text figures in all other
%doc%  circumstances.'
%doc% \end{quote}
%doc% 
%doc% \paragraph{How?} 
%doc% Quoted from Wikibooks\footnote{\url{https://secure.wikimedia.org/wikibooks/en/wiki/LaTeX/Formatting\#Text\_figures\_.28.22old\_style.22\_numerals.29}}:
%doc% \begin{quote}
%doc% Some fonts do not have text figures built in; the textcomp package attempts to
%doc% remedy this by effectively generating text figures from the currently-selected
%doc% font. Put \verb#\usepackage{textcomp}# in your preamble. textcomp also allows you to
%doc% use decimal points, properly formatted dollar signs, etc. within
%doc% \verb#\oldstylenums{}#.
%doc% \end{quote}
%doc% \ldots but proposed \LaTeX{} method does not work out well. Instead use:\\
%doc% \verb#\usepackage{hfoldsty}#  (enables text figures using additional font) or \\
%doc% \verb#\usepackage[sc,osf]{mathpazo}# (switches to Palatino font with small caps and old style figures enabled).
%doc%
%\usepackage{hfoldsty}  %% enables text figures using additional font
%% ... OR use ...
\usepackage[sc,osf]{mathpazo} %% switches to Palatino with small caps and old style figures

%% Font selection from:
%%     http://www.matthiaspospiech.de/latex/vorlagen/allgemein/preambel/fonts/
%% use following lines *instead* of the mathpazo package above:
%% ===== Serif =========================================================
%% for Computer Modern (LaTeX default font), simply remove the mathpazo above
%\usepackage{charter}\linespread{1.05} %% Charter
%\usepackage{bookman}                  %% Bookman (laedt Avant Garde !!)
%\usepackage{newcent}                  %% New Century Schoolbook (laedt Avant Garde !!)
%% ===== Sans Serif ====================================================
%\renewcommand{\familydefault}{\sfdefault}  %% this one in *combination* with the default mathpazo package
%\usepackage{cmbright}                  %% CM-Bright (eigntlich eine Familie)
%\usepackage{tpslifonts}                %% tpslifonts % Font for Slides


%doc% 
%doc% \subsection{\texttt{myacro} --- Abbrevations using \textsc{small caps}}\myinteresting
%doc% \label{sec:myacro}
%doc% 
%doc% \paragraph{Why?} see~\textcite[p.\,45f]{Bringhurst1993}: `3.2.2 For abbrevations and
%doc% acronyms in the midst of normal text, use spaced small caps.'
%doc% 
%doc% \paragraph{How?} Using the predefined macro \verb#\myacro{}# for things like
%doc% \myacro{UNO} or \myacro{UNESCO} using \verb#\myacro{UNO}# or \verb#\myacro{UNESCO}#.
%doc% 
\DeclareRobustCommand{\myacro}[1]{\textsc{\lowercase{#1}}} %%  abbrevations using small caps


%doc% 
%doc% \subsection{Colorized headings and links}
%doc% 
%doc% This document template is able to generate an output that uses colorized
%doc% headings, captions, page numbers, and links. The color named `DispositionColor'
%doc% used in this document is defined near the definition of package \texttt{color}
%doc% in the preamble (see section~\ref{subsec:miscpackages}). The changes required
%doc% for headings, page numbers, and captions are defined here.
%doc% 
%doc% Settings for colored links are handled by the definitions of the
%doc% \texttt{hyperref} package (see section~\ref{sec:pdf}).
%doc% 
\setheadsepline{.4pt}[\color{DispositionColor}]
\renewcommand{\headfont}{\normalfont\sffamily\color{DispositionColor}}
\renewcommand{\pnumfont}{\normalfont\sffamily\color{DispositionColor}}
\addtokomafont{disposition}{\color{DispositionColor}}
\addtokomafont{caption}{\color{DispositionColor}\footnotesize}
\addtokomafont{captionlabel}{\color{DispositionColor}}

%doc% 
%doc% \subsection{No figures or tables below footnotes}
%doc% 
%doc% \LaTeX{} places floating environments below footnotes if \texttt{b}
%doc% (bottom) is used as (default) placement algorithm. This is certainly
%doc% not appealing for most people and is deactivated in this template by
%doc% using the package \texttt{footmisc} with its option \texttt{bottom}.
%doc% 
%% see also: http://www.komascript.de/node/858 (German description)
\usepackage[bottom]{footmisc}

%doc% 
%doc% \subsection{Spacings of list environments}
%doc% 
%doc% By default, \LaTeX{} is using vertical spaces between items of enumerate, 
%doc% itemize and description environments. This is fine for multi-line items.
%doc% Many times, the user does just write single-line items where the larger
%doc% vertical space is inappropriate. The \href{http://ctan.org/pkg/enumitem}{enumitem}
%doc% package provides replacements for the pre-defined list environments and
%doc% offers many options to modify their appearances.
%doc% This template is using the package option for \texttt{noitemsep} which
%doc% mimimizes the vertical space between list items.
%doc% 
\usepackage{enumitem}
\setlist{noitemsep}   %% kills the space between items

%doc% 
%doc% \subsection{\texttt{csquotes} --- Correct quotation marks}\myinteresting
%doc% \label{sub:csquotes}
%doc% 
%doc% \emph{Never} use quotation marks found on your keyboard.
%doc% They end up in strange characters or false looking quotation marks.
%doc% 
%doc% In \LaTeX{} you are able to use typographically correct quotation marks. The package 
%doc% \href{http://www.ctan.org/pkg/csquotes}{\texttt{csquotes}} offers you with 
%doc% \verb#\enquote{foobar}# a command to get correct quotation marks around \enquote{foobar}.
%doc% Please do check the package options in order to modify
%doc% its settings according to the language used\footnote{most of the time in 
%doc% combination with the language set in the options of the \texttt{babel} package}.
%doc% 
%doc% \href{http://www.ctan.org/pkg/csquotes}{\texttt{csquotes}} is also recommended 
%doc% by \texttt{biblatex} (see Section~\ref{sec:references}). 
\usepackage[babel=true,strict=true,english=american,german=guillemets]{csquotes}

%doc% 
%doc% \subsection{Line spread}
%doc% 
%doc% If you have to enlarge the distance between two lines of text, you can
%doc% increase it using the \texttt{\mylinespread} command in \texttt{main.tex}. By default, it is
%doc% deactivated (set to 100~percent). Modify only with caution since it influences the
%doc% page layout and could lead to ugly looking documents.
\linespread{\mylinespread}

%doc% 
%doc% \subsection{Optional: Lines above and below the chapter head}
%doc% 
%doc% This is not quite something typographic but rather a matter of taste.
%doc% \myacro{KOMA} Script offers \href{http://www.komascript.de/node/24}{a method to
%doc% add lines above and below chapter head} which is disabled by
%doc% default. If you want to enable this feature, remove corresponding
%doc% comment characters from the settings.
%doc% 
%% Source: http://www.komascript.de/node/24
%disabled% %% 1st get a new command
%disabled% \newcommand*{\ORIGchapterheadstartvskip}{}%
%disabled% %% 2nd save the original definition to the new command
%disabled% \let\ORIGchapterheadstartvskip=\chapterheadstartvskip
%disabled% %% 3rd redefine the command using the saved original command
%disabled% \renewcommand*{\chapterheadstartvskip}{%
%disabled%   \ORIGchapterheadstartvskip
%disabled%   {%
%disabled%     \setlength{\parskip}{0pt}%
%disabled%     \noindent\color{DispositionColor}\rule[.3\baselineskip]{\linewidth}{1pt}\par
%disabled%   }%
%disabled% }
%disabled% %% see above
%disabled% \newcommand*{\ORIGchapterheadendvskip}{}%
%disabled% \let\ORIGchapterheadendvskip=\chapterheadendvskip
%disabled% \renewcommand*{\chapterheadendvskip}{%
%disabled%   {%
%disabled%     \setlength{\parskip}{0pt}%
%disabled%     \noindent\color{DispositionColor}\rule[.3\baselineskip]{\linewidth}{1pt}\par
%disabled%   }%
%disabled%   \ORIGchapterheadendvskip
%disabled% }

%doc% 
%doc% \subsection{Optional: Chapter thumbs}
%doc% 
%doc% This is not quite something typographic but rather a matter of taste.
%doc% \myacro{KOMA} Script offers \href{http://www.komascript.de/chapterthumbs-example}{a method to
%doc% add chapter thumbs} (in combination with the package \texttt{scrpage2}) which is disabled by
%doc% default. If you want to enable this feature, remove corresponding
%doc% comment characters from the settings.
%doc% 
%disabled% \makeatletter
%disabled% % Safty first
%disabled% \@ifundefined{chapter}{\let\chapter\undefined
%disabled%   \chapter must be defined to use chapter thumbs!}{%
%disabled%  
%disabled% % Two new commands for the width and height of the boxes with the
%disabled% % chapter number at the thumbs (use of commands instead of lengths
%disabled% % for sparing registers)
%disabled% \newcommand*{\chapterthumbwidth}{2em}
%disabled% \newcommand*{\chapterthumbheight}{1em}
%disabled%  
%disabled% % Two new commands for the colors of the box background and the
%disabled% % chapter numbers of the thumbs
%disabled% \newcommand*{\chapterthumbboxcolor}{black}
%disabled% \newcommand*{\chapterthumbtextcolor}{white}
%disabled%  
%disabled% % New command to set a chapter thumb. I'm using a group at this
%disabled% % command, because I'm changing the temporary dimension \@tempdima
%disabled% \newcommand*{\putchapterthumb}{%
%disabled%   \begingroup
%disabled%     \Large
%disabled%     % calculate the horizontal possition of the right paper border
%disabled%     % (I ignore \hoffset, because I interprete \hoffset moves the page
%disabled%     % at the paper e.g. if you are using cropmarks)
%disabled%     \setlength{\@tempdima}{\@oddheadshift}% (internal from scrpage2)
%disabled%     \setlength{\@tempdima}{-\@tempdima}%
%disabled%     \addtolength{\@tempdima}{\paperwidth}%
%disabled%     \addtolength{\@tempdima}{-\oddsidemargin}%
%disabled%     \addtolength{\@tempdima}{-1in}%
%disabled%     % putting the thumbs should not change the horizontal
%disabled%     % possition
%disabled%     \rlap{%
%disabled%       % move to the calculated horizontal possition
%disabled%       \hspace*{\@tempdima}%
%disabled%       % putting the thumbs should not change the vertical
%disabled%       % possition
%disabled%       \vbox to 0pt{%
%disabled%         % calculate the vertical possition of the thumbs (I ignore
%disabled%         % \voffset for the same reasons told above)
%disabled%         \setlength{\@tempdima}{\chapterthumbwidth}%
%disabled%         \multiply\@tempdima by\value{chapter}%
%disabled%         \addtolength{\@tempdima}{-\chapterthumbwidth}%
%disabled%         \addtolength{\@tempdima}{-\baselineskip}%
%disabled%         % move to the calculated vertical possition
%disabled%         \vspace*{\@tempdima}%
%disabled%         % put the thumbs left so the current horizontal possition
%disabled%         \llap{%
%disabled%           % and rotate them
%disabled%           \rotatebox{90}{\colorbox{\chapterthumbboxcolor}{%
%disabled%               \parbox[c][\chapterthumbheight][c]{\chapterthumbwidth}{%
%disabled%                 \centering
%disabled%                 \textcolor{\chapterthumbtextcolor}{%
%disabled%                   \strut\thechapter}\\
%disabled%               }%
%disabled%             }%
%disabled%           }%
%disabled%         }%
%disabled%         % avoid overfull \vbox messages
%disabled%         \vss
%disabled%       }%
%disabled%     }%
%disabled%   \endgroup
%disabled% }
%disabled%  
%disabled% % New command, which works like \lohead but also puts the thumbs (you
%disabled% % cannot use \ihead with this definition but you may change this, if
%disabled% % you use more internal scrpage2 commands)
%disabled% \newcommand*{\loheadwithchapterthumbs}[2][]{%
%disabled%   \lohead[\putchapterthumb#1]{\putchapterthumb#2}%
%disabled% }
%disabled%  
%disabled% % initial use
%disabled% \loheadwithchapterthumbs{}
%disabled% \pagestyle{scrheadings}
%disabled%  
%disabled% }
%disabled% \makeatother

%%%% END
%%% Local Variables:
%%% mode: latex
%%% mode: auto-fill
%%% mode: flyspell
%%% eval: (ispell-change-dictionary "en_US")
%%% TeX-master: "../main"
%%% End:
%% vim:foldmethod=expr
%% vim:fde=getline(v\:lnum)=~'^%%%%'?0\:getline(v\:lnum)=~'^%doc.*\ .\\%(sub\\)\\?section{.\\+'?'>1'\:'1':
# command
%doc% in \texttt{main.tex}.  For standard usage it is recommended to stay with the
%doc% default settings.
%doc% 
%doc% 
%% ========================================================================

%doc%
%doc% Some basic microtypographic settings are provided by the
%doc% \texttt{microtype}
%doc% package\footnote{\url{http://ctan.org/pkg/microtype}}. This template
%doc% uses the rather conservative package parameters: \texttt{protrusion=true,factor=900}.
\usepackage[protrusion=true,factor=900]{microtype}

%doc%
%doc% \subsection{French spacing}
%doc%
%doc% \paragraph{Why?} see~\textcite[p.\,28, p.\,30]{Bringhurst1993}: `2.1.4 Use a single word space between sentences.'
%doc%
%doc% \paragraph{How?} see~\textcite[p.\,185]{Eijkhout2008}:\\
%doc% \verb#\frenchspacing  %% Macro to switch off extra space after punctuation.# \\
\frenchspacing  %% Macro to switch off extra space after punctuation.
%doc%
%doc% Note: This setting might be default for \myacro{KOMA} script.
%doc%


%doc%
%doc% \subsection{Font}
%doc% 
%doc% This template is using the Palatino font (package \texttt{mathpazo}) which results
%doc% in a legible document and matching mathematical fonts for printout.
%doc% 
%doc% It is highly recommended that you either stick to the Palatino font or use the
%doc% \LaTeX{} default fonts (by removing the package \texttt{mathpazo}).
%doc% 
%doc% Chosing different fonts is not
%doc% an easy task. Please leave this to people with good knowledge on this subject.
%doc% 
%doc% One valid reason to change the default fonts is when your document is mainly
%doc% read on a computer screen. In this case it is recommended to switch to a font
%doc% \textsf{which is sans-serif like this}. This template contains several alternative
%doc% font packages which can be activated in this file.
%doc% 

% for changing the default font, please go to the next subsection!

%doc%
%doc% \subsection{Text figures}
%doc% 
%doc% \ldots also called old style numbers such as 0123456789. 
%doc% (German: \enquote{Mediäval\-ziffern\footnote{\url{https://secure.wikimedia.org/wikibooks/de/wiki/LaTeX-W\%C3\%B6rterbuch:\_Medi\%C3\%A4valziffern}}})
%doc% \paragraph{Why?} see~\textcite[p.\,44f]{Bringhurst1993}: 
%doc% \begin{quote}
%doc% `3.2.1 If the font includes both text figures and titling figures, use
%doc%  titling figures only with full caps, and text figures in all other
%doc%  circumstances.'
%doc% \end{quote}
%doc% 
%doc% \paragraph{How?} 
%doc% Quoted from Wikibooks\footnote{\url{https://secure.wikimedia.org/wikibooks/en/wiki/LaTeX/Formatting\#Text\_figures\_.28.22old\_style.22\_numerals.29}}:
%doc% \begin{quote}
%doc% Some fonts do not have text figures built in; the textcomp package attempts to
%doc% remedy this by effectively generating text figures from the currently-selected
%doc% font. Put \verb#\usepackage{textcomp}# in your preamble. textcomp also allows you to
%doc% use decimal points, properly formatted dollar signs, etc. within
%doc% \verb#\oldstylenums{}#.
%doc% \end{quote}
%doc% \ldots but proposed \LaTeX{} method does not work out well. Instead use:\\
%doc% \verb#\usepackage{hfoldsty}#  (enables text figures using additional font) or \\
%doc% \verb#\usepackage[sc,osf]{mathpazo}# (switches to Palatino font with small caps and old style figures enabled).
%doc%
%\usepackage{hfoldsty}  %% enables text figures using additional font
%% ... OR use ...
\usepackage[sc,osf]{mathpazo} %% switches to Palatino with small caps and old style figures

%% Font selection from:
%%     http://www.matthiaspospiech.de/latex/vorlagen/allgemein/preambel/fonts/
%% use following lines *instead* of the mathpazo package above:
%% ===== Serif =========================================================
%% for Computer Modern (LaTeX default font), simply remove the mathpazo above
%\usepackage{charter}\linespread{1.05} %% Charter
%\usepackage{bookman}                  %% Bookman (laedt Avant Garde !!)
%\usepackage{newcent}                  %% New Century Schoolbook (laedt Avant Garde !!)
%% ===== Sans Serif ====================================================
%\renewcommand{\familydefault}{\sfdefault}  %% this one in *combination* with the default mathpazo package
%\usepackage{cmbright}                  %% CM-Bright (eigntlich eine Familie)
%\usepackage{tpslifonts}                %% tpslifonts % Font for Slides


%doc% 
%doc% \subsection{\texttt{myacro} --- Abbrevations using \textsc{small caps}}\myinteresting
%doc% \label{sec:myacro}
%doc% 
%doc% \paragraph{Why?} see~\textcite[p.\,45f]{Bringhurst1993}: `3.2.2 For abbrevations and
%doc% acronyms in the midst of normal text, use spaced small caps.'
%doc% 
%doc% \paragraph{How?} Using the predefined macro \verb#\myacro{}# for things like
%doc% \myacro{UNO} or \myacro{UNESCO} using \verb#\myacro{UNO}# or \verb#\myacro{UNESCO}#.
%doc% 
\DeclareRobustCommand{\myacro}[1]{\textsc{\lowercase{#1}}} %%  abbrevations using small caps


%doc% 
%doc% \subsection{Colorized headings and links}
%doc% 
%doc% This document template is able to generate an output that uses colorized
%doc% headings, captions, page numbers, and links. The color named `DispositionColor'
%doc% used in this document is defined near the definition of package \texttt{color}
%doc% in the preamble (see section~\ref{subsec:miscpackages}). The changes required
%doc% for headings, page numbers, and captions are defined here.
%doc% 
%doc% Settings for colored links are handled by the definitions of the
%doc% \texttt{hyperref} package (see section~\ref{sec:pdf}).
%doc% 
\setheadsepline{.4pt}[\color{DispositionColor}]
\renewcommand{\headfont}{\normalfont\sffamily\color{DispositionColor}}
\renewcommand{\pnumfont}{\normalfont\sffamily\color{DispositionColor}}
\addtokomafont{disposition}{\color{DispositionColor}}
\addtokomafont{caption}{\color{DispositionColor}\footnotesize}
\addtokomafont{captionlabel}{\color{DispositionColor}}

%doc% 
%doc% \subsection{No figures or tables below footnotes}
%doc% 
%doc% \LaTeX{} places floating environments below footnotes if \texttt{b}
%doc% (bottom) is used as (default) placement algorithm. This is certainly
%doc% not appealing for most people and is deactivated in this template by
%doc% using the package \texttt{footmisc} with its option \texttt{bottom}.
%doc% 
%% see also: http://www.komascript.de/node/858 (German description)
\usepackage[bottom]{footmisc}

%doc% 
%doc% \subsection{Spacings of list environments}
%doc% 
%doc% By default, \LaTeX{} is using vertical spaces between items of enumerate, 
%doc% itemize and description environments. This is fine for multi-line items.
%doc% Many times, the user does just write single-line items where the larger
%doc% vertical space is inappropriate. The \href{http://ctan.org/pkg/enumitem}{enumitem}
%doc% package provides replacements for the pre-defined list environments and
%doc% offers many options to modify their appearances.
%doc% This template is using the package option for \texttt{noitemsep} which
%doc% mimimizes the vertical space between list items.
%doc% 
\usepackage{enumitem}
\setlist{noitemsep}   %% kills the space between items

%doc% 
%doc% \subsection{\texttt{csquotes} --- Correct quotation marks}\myinteresting
%doc% \label{sub:csquotes}
%doc% 
%doc% \emph{Never} use quotation marks found on your keyboard.
%doc% They end up in strange characters or false looking quotation marks.
%doc% 
%doc% In \LaTeX{} you are able to use typographically correct quotation marks. The package 
%doc% \href{http://www.ctan.org/pkg/csquotes}{\texttt{csquotes}} offers you with 
%doc% \verb#\enquote{foobar}# a command to get correct quotation marks around \enquote{foobar}.
%doc% Please do check the package options in order to modify
%doc% its settings according to the language used\footnote{most of the time in 
%doc% combination with the language set in the options of the \texttt{babel} package}.
%doc% 
%doc% \href{http://www.ctan.org/pkg/csquotes}{\texttt{csquotes}} is also recommended 
%doc% by \texttt{biblatex} (see Section~\ref{sec:references}). 
\usepackage[babel=true,strict=true,english=american,german=guillemets]{csquotes}

%doc% 
%doc% \subsection{Line spread}
%doc% 
%doc% If you have to enlarge the distance between two lines of text, you can
%doc% increase it using the \texttt{\mylinespread} command in \texttt{main.tex}. By default, it is
%doc% deactivated (set to 100~percent). Modify only with caution since it influences the
%doc% page layout and could lead to ugly looking documents.
\linespread{\mylinespread}

%doc% 
%doc% \subsection{Optional: Lines above and below the chapter head}
%doc% 
%doc% This is not quite something typographic but rather a matter of taste.
%doc% \myacro{KOMA} Script offers \href{http://www.komascript.de/node/24}{a method to
%doc% add lines above and below chapter head} which is disabled by
%doc% default. If you want to enable this feature, remove corresponding
%doc% comment characters from the settings.
%doc% 
%% Source: http://www.komascript.de/node/24
%disabled% %% 1st get a new command
%disabled% \newcommand*{\ORIGchapterheadstartvskip}{}%
%disabled% %% 2nd save the original definition to the new command
%disabled% \let\ORIGchapterheadstartvskip=\chapterheadstartvskip
%disabled% %% 3rd redefine the command using the saved original command
%disabled% \renewcommand*{\chapterheadstartvskip}{%
%disabled%   \ORIGchapterheadstartvskip
%disabled%   {%
%disabled%     \setlength{\parskip}{0pt}%
%disabled%     \noindent\color{DispositionColor}\rule[.3\baselineskip]{\linewidth}{1pt}\par
%disabled%   }%
%disabled% }
%disabled% %% see above
%disabled% \newcommand*{\ORIGchapterheadendvskip}{}%
%disabled% \let\ORIGchapterheadendvskip=\chapterheadendvskip
%disabled% \renewcommand*{\chapterheadendvskip}{%
%disabled%   {%
%disabled%     \setlength{\parskip}{0pt}%
%disabled%     \noindent\color{DispositionColor}\rule[.3\baselineskip]{\linewidth}{1pt}\par
%disabled%   }%
%disabled%   \ORIGchapterheadendvskip
%disabled% }

%doc% 
%doc% \subsection{Optional: Chapter thumbs}
%doc% 
%doc% This is not quite something typographic but rather a matter of taste.
%doc% \myacro{KOMA} Script offers \href{http://www.komascript.de/chapterthumbs-example}{a method to
%doc% add chapter thumbs} (in combination with the package \texttt{scrpage2}) which is disabled by
%doc% default. If you want to enable this feature, remove corresponding
%doc% comment characters from the settings.
%doc% 
%disabled% \makeatletter
%disabled% % Safty first
%disabled% \@ifundefined{chapter}{\let\chapter\undefined
%disabled%   \chapter must be defined to use chapter thumbs!}{%
%disabled%  
%disabled% % Two new commands for the width and height of the boxes with the
%disabled% % chapter number at the thumbs (use of commands instead of lengths
%disabled% % for sparing registers)
%disabled% \newcommand*{\chapterthumbwidth}{2em}
%disabled% \newcommand*{\chapterthumbheight}{1em}
%disabled%  
%disabled% % Two new commands for the colors of the box background and the
%disabled% % chapter numbers of the thumbs
%disabled% \newcommand*{\chapterthumbboxcolor}{black}
%disabled% \newcommand*{\chapterthumbtextcolor}{white}
%disabled%  
%disabled% % New command to set a chapter thumb. I'm using a group at this
%disabled% % command, because I'm changing the temporary dimension \@tempdima
%disabled% \newcommand*{\putchapterthumb}{%
%disabled%   \begingroup
%disabled%     \Large
%disabled%     % calculate the horizontal possition of the right paper border
%disabled%     % (I ignore \hoffset, because I interprete \hoffset moves the page
%disabled%     % at the paper e.g. if you are using cropmarks)
%disabled%     \setlength{\@tempdima}{\@oddheadshift}% (internal from scrpage2)
%disabled%     \setlength{\@tempdima}{-\@tempdima}%
%disabled%     \addtolength{\@tempdima}{\paperwidth}%
%disabled%     \addtolength{\@tempdima}{-\oddsidemargin}%
%disabled%     \addtolength{\@tempdima}{-1in}%
%disabled%     % putting the thumbs should not change the horizontal
%disabled%     % possition
%disabled%     \rlap{%
%disabled%       % move to the calculated horizontal possition
%disabled%       \hspace*{\@tempdima}%
%disabled%       % putting the thumbs should not change the vertical
%disabled%       % possition
%disabled%       \vbox to 0pt{%
%disabled%         % calculate the vertical possition of the thumbs (I ignore
%disabled%         % \voffset for the same reasons told above)
%disabled%         \setlength{\@tempdima}{\chapterthumbwidth}%
%disabled%         \multiply\@tempdima by\value{chapter}%
%disabled%         \addtolength{\@tempdima}{-\chapterthumbwidth}%
%disabled%         \addtolength{\@tempdima}{-\baselineskip}%
%disabled%         % move to the calculated vertical possition
%disabled%         \vspace*{\@tempdima}%
%disabled%         % put the thumbs left so the current horizontal possition
%disabled%         \llap{%
%disabled%           % and rotate them
%disabled%           \rotatebox{90}{\colorbox{\chapterthumbboxcolor}{%
%disabled%               \parbox[c][\chapterthumbheight][c]{\chapterthumbwidth}{%
%disabled%                 \centering
%disabled%                 \textcolor{\chapterthumbtextcolor}{%
%disabled%                   \strut\thechapter}\\
%disabled%               }%
%disabled%             }%
%disabled%           }%
%disabled%         }%
%disabled%         % avoid overfull \vbox messages
%disabled%         \vss
%disabled%       }%
%disabled%     }%
%disabled%   \endgroup
%disabled% }
%disabled%  
%disabled% % New command, which works like \lohead but also puts the thumbs (you
%disabled% % cannot use \ihead with this definition but you may change this, if
%disabled% % you use more internal scrpage2 commands)
%disabled% \newcommand*{\loheadwithchapterthumbs}[2][]{%
%disabled%   \lohead[\putchapterthumb#1]{\putchapterthumb#2}%
%disabled% }
%disabled%  
%disabled% % initial use
%disabled% \loheadwithchapterthumbs{}
%disabled% \pagestyle{scrheadings}
%disabled%  
%disabled% }
%disabled% \makeatother

%%%% END
%%% Local Variables:
%%% mode: latex
%%% mode: auto-fill
%%% mode: flyspell
%%% eval: (ispell-change-dictionary "en_US")
%%% TeX-master: "../main"
%%% End:
%% vim:foldmethod=expr
%% vim:fde=getline(v\:lnum)=~'^%%%%'?0\:getline(v\:lnum)=~'^%doc.*\ .\\%(sub\\)\\?section{.\\+'?'>1'\:'1':
# command
%doc% in \texttt{main.tex}.  For standard usage it is recommended to stay with the
%doc% default settings.
%doc%
%doc%
%% ========================================================================

%doc%
%doc% Some basic microtypographic settings are provided by the
%doc% \texttt{microtype}
%doc% package\footnote{\url{http://ctan.org/pkg/microtype}}. This template
%doc% uses the rather conservative package parameters: \texttt{protrusion=true,factor=900}.
\usepackage[protrusion=true,factor=900]{microtype}

%doc%
%doc% \subsection{French spacing}
%doc%
%doc% \paragraph{Why?} see~\textcite[p.\,28, p.\,30]{Bringhurst1993}: `2.1.4 Use a single word space between sentences.'
%doc%
%doc% \paragraph{How?} see~\textcite[p.\,185]{Eijkhout2008}:\\
%doc% \verb#\frenchspacing  %% Macro to switch off extra space after punctuation.# \\
\frenchspacing  %% Macro to switch off extra space after punctuation.
%doc%
%doc% Note: This setting might be default for \myacro{KOMA} script.
%doc%


%doc%
%doc% \subsection{Font}
%doc%
%doc% This template is using the Palatino font (package \texttt{mathpazo}) which results
%doc% in a legible document and matching mathematical fonts for printout.
%doc%
%doc% It is highly recommended that you either stick to the Palatino font or use the
%doc% \LaTeX{} default fonts (by removing the package \texttt{mathpazo}).
%doc%
%doc% Chosing different fonts is not
%doc% an easy task. Please leave this to people with good knowledge on this subject.
%doc%
%doc% One valid reason to change the default fonts is when your document is mainly
%doc% read on a computer screen. In this case it is recommended to switch to a font
%doc% \textsf{which is sans-serif like this}. This template contains several alternative
%doc% font packages which can be activated in this file.
%doc%

% for changing the default font, please go to the next subsection!

%doc%
%doc% \subsection{Text figures}
%doc%
%doc% \ldots also called old style numbers such as 0123456789.
%doc% (German: \enquote{Mediäval\-ziffern\footnote{\url{https://secure.wikimedia.org/wikibooks/de/wiki/LaTeX-W\%C3\%B6rterbuch:\_Medi\%C3\%A4valziffern}}})
%doc% \paragraph{Why?} see~\textcite[p.\,44f]{Bringhurst1993}:
%doc% \begin{quote}
%doc% `3.2.1 If the font includes both text figures and titling figures, use
%doc%  titling figures only with full caps, and text figures in all other
%doc%  circumstances.'
%doc% \end{quote}
%doc%
%doc% \paragraph{How?}
%doc% Quoted from Wikibooks\footnote{\url{https://secure.wikimedia.org/wikibooks/en/wiki/LaTeX/Formatting\#Text\_figures\_.28.22old\_style.22\_numerals.29}}:
%doc% \begin{quote}
%doc% Some fonts do not have text figures built in; the textcomp package attempts to
%doc% remedy this by effectively generating text figures from the currently-selected
%doc% font. Put \verb#\usepackage{textcomp}# in your preamble. textcomp also allows you to
%doc% use decimal points, properly formatted dollar signs, etc. within
%doc% \verb#\oldstylenums{}#.
%doc% \end{quote}
%doc% \ldots but proposed \LaTeX{} method does not work out well. Instead use:\\
%doc% \verb#\usepackage{hfoldsty}#  (enables text figures using additional font) or \\
%doc% \verb#\usepackage[sc,osf]{mathpazo}# (switches to Palatino font with small caps and old style figures enabled).
%doc%
%\usepackage{hfoldsty}  %% enables text figures using additional font
%% ... OR use ...
\usepackage[sc,osf]{mathpazo} %% switches to Palatino with small caps and old style figures

%% Font selection from:
%%     http://www.matthiaspospiech.de/latex/vorlagen/allgemein/preambel/fonts/
%% use following lines *instead* of the mathpazo package above:
%% ===== Serif =========================================================
%% for Computer Modern (LaTeX default font), simply remove the mathpazo above
%\usepackage{charter}\linespread{1.05} %% Charter
%\usepackage{bookman}                  %% Bookman (laedt Avant Garde !!)
%\usepackage{newcent}                  %% New Century Schoolbook (laedt Avant Garde !!)
%% ===== Sans Serif ====================================================
%\renewcommand{\familydefault}{\sfdefault}  %% this one in *combination* with the default mathpazo package
%\usepackage{cmbright}                  %% CM-Bright (eigntlich eine Familie)
%\usepackage{tpslifonts}                %% tpslifonts % Font for Slides


%doc%
%doc% \subsection{\texttt{myacro} --- Abbrevations using \textsc{small caps}}\myinteresting
%doc% \label{sec:myacro}
%doc%
%doc% \paragraph{Why?} see~\textcite[p.\,45f]{Bringhurst1993}: `3.2.2 For abbrevations and
%doc% acronyms in the midst of normal text, use spaced small caps.'
%doc%
%doc% \paragraph{How?} Using the predefined macro \verb#\myacro{}# for things like
%doc% \myacro{UNO} or \myacro{UNESCO} using \verb#\myacro{UNO}# or \verb#\myacro{UNESCO}#.
%doc%
\DeclareRobustCommand{\myacro}[1]{\textsc{\lowercase{#1}}} %%  abbrevations using small caps


%doc%
%doc% \subsection{Colorized headings and links}
%doc%
%doc% This document template is able to generate an output that uses colorized
%doc% headings, captions, page numbers, and links. The color named `DispositionColor'
%doc% used in this document is defined near the definition of package \texttt{color}
%doc% in the preamble (see section~\ref{subsec:miscpackages}). The changes required
%doc% for headings, page numbers, and captions are defined here.
%doc%
%doc% Settings for colored links are handled by the definitions of the
%doc% \texttt{hyperref} package (see section~\ref{sec:pdf}).
%doc%
\setheadsepline{.4pt}[\color{DispositionColor}]
\renewcommand{\headfont}{\normalfont\sffamily\color{DispositionColor}}
\renewcommand{\pnumfont}{\normalfont\sffamily\color{DispositionColor}}
\addtokomafont{disposition}{\color{DispositionColor}}
\addtokomafont{caption}{\color{DispositionColor}\footnotesize}
\addtokomafont{captionlabel}{\color{DispositionColor}}

%doc%
%doc% \subsection{No figures or tables below footnotes}
%doc%
%doc% \LaTeX{} places floating environments below footnotes if \texttt{b}
%doc% (bottom) is used as (default) placement algorithm. This is certainly
%doc% not appealing for most people and is deactivated in this template by
%doc% using the package \texttt{footmisc} with its option \texttt{bottom}.
%doc%
%% see also: http://www.komascript.de/node/858 (German description)
\usepackage[bottom]{footmisc}

%doc%
%doc% \subsection{Spacings of list environments}
%doc%
%doc% By default, \LaTeX{} is using vertical spaces between items of enumerate,
%doc% itemize and description environments. This is fine for multi-line items.
%doc% Many times, the user does just write single-line items where the larger
%doc% vertical space is inappropriate. The \href{http://ctan.org/pkg/enumitem}{enumitem}
%doc% package provides replacements for the pre-defined list environments and
%doc% offers many options to modify their appearances.
%doc% This template is using the package option for \texttt{noitemsep} which
%doc% mimimizes the vertical space between list items.
%doc%
\usepackage{enumitem}
\setlist{noitemsep}   %% kills the space between items

%doc%
%doc% \subsection{\texttt{csquotes} --- Correct quotation marks}\myinteresting
%doc% \label{sub:csquotes}
%doc%
%doc% \emph{Never} use quotation marks found on your keyboard.
%doc% They end up in strange characters or false looking quotation marks.
%doc%
%doc% In \LaTeX{} you are able to use typographically correct quotation marks. The package
%doc% \href{http://www.ctan.org/pkg/csquotes}{\texttt{csquotes}} offers you with
%doc% \verb#\enquote{foobar}# a command to get correct quotation marks around \enquote{foobar}.
%doc% Please do check the package options in order to modify
%doc% its settings according to the language used\footnote{most of the time in
%doc% combination with the language set in the options of the \texttt{babel} package}.
%doc%
%doc% \href{http://www.ctan.org/pkg/csquotes}{\texttt{csquotes}} is also recommended
%doc% by \texttt{biblatex} (see Section~\ref{sec:references}).
\usepackage[babel=true,strict=true,english=american,german=guillemets]{csquotes}

%doc%
%doc% \subsection{Line spread}
%doc%
%doc% If you have to enlarge the distance between two lines of text, you can
%doc% increase it using the \texttt{\mylinespread} command in \texttt{main.tex}. By default, it is
%doc% deactivated (set to 100~percent). Modify only with caution since it influences the
%doc% page layout and could lead to ugly looking documents.
\linespread{\mylinespread}

%doc%
%doc% \subsection{Optional: Lines above and below the chapter head}
%doc%
%doc% This is not quite something typographic but rather a matter of taste.
%doc% \myacro{KOMA} Script offers \href{http://www.komascript.de/node/24}{a method to
%doc% add lines above and below chapter head} which is disabled by
%doc% default. If you want to enable this feature, remove corresponding
%doc% comment characters from the settings.
%doc%
%% Source: http://www.komascript.de/node/24
%disabled% %% 1st get a new command
%disabled% \newcommand*{\ORIGchapterheadstartvskip}{}%
%disabled% %% 2nd save the original definition to the new command
%disabled% \let\ORIGchapterheadstartvskip=\chapterheadstartvskip
%disabled% %% 3rd redefine the command using the saved original command
%disabled% \renewcommand*{\chapterheadstartvskip}{%
%disabled%   \ORIGchapterheadstartvskip
%disabled%   {%
%disabled%     \setlength{\parskip}{0pt}%
%disabled%     \noindent\color{DispositionColor}\rule[.3\baselineskip]{\linewidth}{1pt}\par
%disabled%   }%
%disabled% }
%disabled% %% see above
%disabled% \newcommand*{\ORIGchapterheadendvskip}{}%
%disabled% \let\ORIGchapterheadendvskip=\chapterheadendvskip
%disabled% \renewcommand*{\chapterheadendvskip}{%
%disabled%   {%
%disabled%     \setlength{\parskip}{0pt}%
%disabled%     \noindent\color{DispositionColor}\rule[.3\baselineskip]{\linewidth}{1pt}\par
%disabled%   }%
%disabled%   \ORIGchapterheadendvskip
%disabled% }

%doc%
%doc% \subsection{Optional: Chapter thumbs}
%doc%
%doc% This is not quite something typographic but rather a matter of taste.
%doc% \myacro{KOMA} Script offers \href{http://www.komascript.de/chapterthumbs-example}{a method to
%doc% add chapter thumbs} (in combination with the package \texttt{scrpage2}) which is disabled by
%doc% default. If you want to enable this feature, remove corresponding
%doc% comment characters from the settings.
%doc%
%disabled% \makeatletter
%disabled% % Safty first
%disabled% \@ifundefined{chapter}{\let\chapter\undefined
%disabled%   \chapter must be defined to use chapter thumbs!}{%
%disabled%
%disabled% % Two new commands for the width and height of the boxes with the
%disabled% % chapter number at the thumbs (use of commands instead of lengths
%disabled% % for sparing registers)
%disabled% \newcommand*{\chapterthumbwidth}{2em}
%disabled% \newcommand*{\chapterthumbheight}{1em}
%disabled%
%disabled% % Two new commands for the colors of the box background and the
%disabled% % chapter numbers of the thumbs
%disabled% \newcommand*{\chapterthumbboxcolor}{black}
%disabled% \newcommand*{\chapterthumbtextcolor}{white}
%disabled%
%disabled% % New command to set a chapter thumb. I'm using a group at this
%disabled% % command, because I'm changing the temporary dimension \@tempdima
%disabled% \newcommand*{\putchapterthumb}{%
%disabled%   \begingroup
%disabled%     \Large
%disabled%     % calculate the horizontal possition of the right paper border
%disabled%     % (I ignore \hoffset, because I interprete \hoffset moves the page
%disabled%     % at the paper e.g. if you are using cropmarks)
%disabled%     \setlength{\@tempdima}{\@oddheadshift}% (internal from scrpage2)
%disabled%     \setlength{\@tempdima}{-\@tempdima}%
%disabled%     \addtolength{\@tempdima}{\paperwidth}%
%disabled%     \addtolength{\@tempdima}{-\oddsidemargin}%
%disabled%     \addtolength{\@tempdima}{-1in}%
%disabled%     % putting the thumbs should not change the horizontal
%disabled%     % possition
%disabled%     \rlap{%
%disabled%       % move to the calculated horizontal possition
%disabled%       \hspace*{\@tempdima}%
%disabled%       % putting the thumbs should not change the vertical
%disabled%       % possition
%disabled%       \vbox to 0pt{%
%disabled%         % calculate the vertical possition of the thumbs (I ignore
%disabled%         % \voffset for the same reasons told above)
%disabled%         \setlength{\@tempdima}{\chapterthumbwidth}%
%disabled%         \multiply\@tempdima by\value{chapter}%
%disabled%         \addtolength{\@tempdima}{-\chapterthumbwidth}%
%disabled%         \addtolength{\@tempdima}{-\baselineskip}%
%disabled%         % move to the calculated vertical possition
%disabled%         \vspace*{\@tempdima}%
%disabled%         % put the thumbs left so the current horizontal possition
%disabled%         \llap{%
%disabled%           % and rotate them
%disabled%           \rotatebox{90}{\colorbox{\chapterthumbboxcolor}{%
%disabled%               \parbox[c][\chapterthumbheight][c]{\chapterthumbwidth}{%
%disabled%                 \centering
%disabled%                 \textcolor{\chapterthumbtextcolor}{%
%disabled%                   \strut\thechapter}\\
%disabled%               }%
%disabled%             }%
%disabled%           }%
%disabled%         }%
%disabled%         % avoid overfull \vbox messages
%disabled%         \vss
%disabled%       }%
%disabled%     }%
%disabled%   \endgroup
%disabled% }
%disabled%
%disabled% % New command, which works like \lohead but also puts the thumbs (you
%disabled% % cannot use \ihead with this definition but you may change this, if
%disabled% % you use more internal scrpage2 commands)
%disabled% \newcommand*{\loheadwithchapterthumbs}[2][]{%
%disabled%   \lohead[\putchapterthumb#1]{\putchapterthumb#2}%
%disabled% }
%disabled%
%disabled% % initial use
%disabled% \loheadwithchapterthumbs{}
%disabled% \pagestyle{scrheadings}
%disabled%
%disabled% }
%disabled% \makeatother

%%%% END
%%% Local Variables:
%%% mode: latex
%%% mode: auto-fill
%%% mode: flyspell
%%% eval: (ispell-change-dictionary "en_US")
%%% TeX-master: "../main"
%%% End:
%% vim:foldmethod=expr
%% vim:fde=getline(v\:lnum)=~'^%%%%'?0\:getline(v\:lnum)=~'^%doc.*\ .\\%(sub\\)\\?section{.\\+'?'>1'\:'1':
